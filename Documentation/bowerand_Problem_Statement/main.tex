\documentclass[letterpaper, 10, draftclsnofoot,onecolumn]{IEEEtran}
\usepackage[utf8]{inputenc}
\usepackage{graphicx}
\usepackage{amssymb}
\usepackage{amsmath}
\usepackage{amsthm}
\usepackage{indentfirst}

\usepackage{alltt}
\usepackage{float}
\usepackage{color}
\usepackage{url}

\usepackage{balance}
\usepackage{enumitem}
\usepackage{geometry}
\geometry{textheight=9.5in, textwidth=7in}


\begin{document}

\title{Problem Statement}
\author
{\IEEEauthorblockN{}
\IEEEauthorblockA{Andrew Bowers\\
CS 461\\
Fall 2017}
}

\maketitle

\begin{abstract}
\begin{center}
 We have a Social Mobile App Journal that needs allow users to make more in-depth posts that followers/friends of said user can like, comment, and even share. The motivation behind this project is to allow users to tell more of a story rather than just make short-generic posts. The project was sponsored by David Vasquez and the group working on it consists of myself, David Baugh, and Sawyer Kokesh.
\end{center}
\end{abstract}

\clearpage
\section*{Introduction} My senior design project is the Social Mobile App Journal (official name to be determined). The project was sponsored by David Vasquez and the group working on it consists of myself, David Baugh, and Sawyer Kokesh. The application will be for iOS, Android, or both (this depends on how we want to implement it). This paper serves to define the project itself, propose solutions as to how we will create it, and performance metrics to gauge our progress as we move through development. 
\section*{Definition and Description} This project is a social network/personal journal in which the user can make larger and more in-depth posts filled with more content and even pictures. Other users or 'friends' of the user who made the original post can see, like, comment, and even share it. The motivation behind this project is to give users the opportunity to make more detailed posts, rather than just small pieces of information. As described by our client, the final product would be something like a cross of Instagram and twitter; it would give the user more freedom in their posts and allow them to tell more of a story rather than just give a snapshot of something going on in their day.
\\ \\ 
\indent The project itself is a mobile application and will consists of a log-in and registration process, user profile, personal journal, and the ability to follow or add other users. For this to happen, we will have to design a database to store user information amongst other relevant data, handle networking, and build the user interface itself. 
\section*{Proposed Solutions} In accomplishing this project, we will need to decide on which device we want to develop the application for, where we want to set up the database, and how we want to build the interface. Starting with deciding which interface (iOS, Android, or both) will most likely be first on our agenda. This will give us an idea on how we will program it as well as what language we'll be using (etc). React Native will allow us to develop the project for both Android and iOS which I feel would be the best decision altogether; it would allow for a lager user base.
\\ \\
\indent As for getting started with the project, I think taking a look at the database side of things would be our best option. It would give us an idea of how everything will be structured as well as how data relates to each other. It'll also inform us of what data is available and can direct how we design the user interface. We can build our database through the school servers or through other third party options (Amazon web hosting etc).
\\ \\
\indent User interface design will be developed to client specification and tested by classmates. We want to make sure the design is fluid and easy to use as well as being enjoyable to look at. To do this, after we've drawn out how we will implement our database, we'll draw out different interfaces and run through how they will work. In doing this we'll be able to find things that don't flow very well or are located in the wrong places on the page. Applying this same dry-run concept (again first on paper) with test users will give us an even better idea of the design for the project.   
\section*{Performance Metrics} Making sure that we are accomplishing tasks in a timely manner is very important to us. To ensure we are on track, we'll be meeting with our TA and client weekly looking over progress as well as the direction we're working in. Our plan as a group is to set out dates in which we can meet up weekly and work for a set period of time on a given task. For example, one of the first things we are looking at working on is the database. Working on it first will give us a bigger idea of how the entire project will look. We'll work all of the relationships out, ensure the tables are in 3rd normal form, and have an idea of what data our users will need to enter (etc). This will allow us to be more informed in our user interface design process as well. Setting several large goals like this then meeting regularly, working on small parts of the larger goals, will allow use to see how far we are coming along. Significant milestones we have so far are the database, networking, user interface, and back-end of the mobile app as well as deciding which operating system we want to use (Android, iOS, or both).
\section*{Conclusion} In conclusion, we have a Social Mobile App Journal that will allow users to make more in-depth posts that followers/friends of said user can like, comment, and even share. The motivation of the project is to allow users to tell more of a story rather than just make short-generic posts. The application name as well as the application operating system is yet to be determined. We will ensure our project is meeting timely goals by meeting with our client and TA weekly as well as making small goals that work towards our larger goals. 
\end{document}
