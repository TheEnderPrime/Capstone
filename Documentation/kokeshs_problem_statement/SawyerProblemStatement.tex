\documentclass[10pt,english,a4paper]{article}
\usepackage{fullpage}
\usepackage[T1]{fontenc}
\usepackage{hyperref}
\usepackage{longtable}
\usepackage{amsmath}

\title{CS461 Problem Statement}
\author{
	Sawyer Kokesh\\
	\texttt{Kokeshs@oregonstate.edu}
	\
}

\begin{document}
\maketitle
	
\section{Project Abstract}
The main purpose of the Social mobile app journal is to provide a space for people to connect with each other in a way that other social media apps don't allow for or they make it difficult. This app will be designed for a user that intends to use it as a journal or blog. In this way it provides a space for users to have longer posts with images spread throughout. Like said previously it is different because the long posts are meant to be, this is different from facebook and applications such as that because it was not designed to take longer posts. Moreover this app will be designed for the user of the journal foremost, the journals are more of a way to tell a story then a game to get the most likes. So this app is more geared toward the individual, but does provide some ways to see and follow other users.
\section{Definition and Description of Problem}
The Social mobile app journal is intended to be an app for a phone to provide users with a new way to connect on social media. Like its predecessors this app will give users the ability to create a profile in which that can customize with personal information about themselves and their journals. Profiles are an integral part of this app because this app will be a platform for users to not just say a few sentences like on Twitter, but provide a place to tell a story. There will be a many types of people that will be using this app that currently have to settle for an app like facebook or twitter or tumblr. Theses users are the ones that want to communicate with other people in a more long form, a more story driven way. These users on other platforms may not be comfortable to write the longer post with may pictures, because in other apps those kinds of posts are �downvoted� or ridiculed in the comment section. There are also the users that are not interested in the short winded stories from other sights and long for a more complete and full story when reading. These are the current problems that are not currently being addressed by a mainstream app, and the purpose of this app is to fill that niche. The niche of full and complete stories that can contain images in an app where this kind of posts are suppose to be so they will not be seen as being a waste of time for other users. Furthermore this app can also be private which will give the user the option to just have a place to write a journal on an app, where they can still follow other people's journals.
\section{Proposed Solution}
The solution to the problems of apps not allowing or making it unappealing for long story driven posts, is to construct a new social mobile journal app that will provide a location that encourages that sort of posts. This new app will be simple and intuitive which will give people a new way to socialize in longer form way. The interface for this app will be similar to other social media apps, for example there will be a homepage or a login page to allow the user to get long into their profile. This app my offer the option to not create an account but, still browse people's journal. This may help get new users to join because it will give potential new users a way to interact with the app without the commitment of signing up for something. Next there will be a profiles screen that will give the user a location to tell other users about themselves and about there journel and what to expect. The journal page will be a scrollable page with newer post at the top and they get older the father you scroll. If the users is on there own journal page there will be a button that they can press to create a new post. This click will open a new page which will be a blank outline for them to fill in with their journal entry and photos if user wants. Also when the user is on there own journal page they will be able to click on old posts, and they be able to edit them. Another page that will be available is a way navigate through or search through other users journals or profiles. 
\section{Performance Metrics}
 The way we will tell if we have successfully completed this project is by showing that we have completed all the the primary requirements for the app. Each requirement is going to be on a hierarchy of Priority, at the first meeting we came up with the base list of deliverables. The main deliverables that we will be providing is one, giving the user a way to register with a username and password. This means we will have to support multiple users and manipulation of the database. The second main deliverable was that each user will have there own journal. This journal will provide the user with space to post their entries. Third, The journal interties or posts can contain both text and images. The images will be able to be formattable within the text and the text will flow around the image. Fourth, Each user will have their own profile. The point of the profile is to show other users a broad summary of what to expect when reading the blog or journal, and information about the author. Fifth, is to give the users the ability to make their journal public or private. When private it can not be followed, or viewable by other users of this app. This leads to the last highest priority deliverable which is allowing users to follow other users journal. 
\end{document}