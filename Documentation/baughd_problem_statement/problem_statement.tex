%% The following template is sourced from Kevin McGrath on 6 Oct. 2017 located:
%%http://web.engr.oregonstate.edu/cgi-bin/cgiwrap/dmcgrath/classes/17F/cs444/index.cgi?examples=examples/template.tex

\documentclass[letterpaper,10pt,titlepage]{article}

\usepackage{graphicx}                                        
\usepackage{amssymb}                                         
\usepackage{amsmath}                                         
\usepackage{amsthm}                                          

\usepackage{alltt}                                           
\usepackage{float}
\usepackage{color}
\usepackage{url}

\usepackage{balance}
\usepackage{enumitem}
\usepackage{pstricks, pst-node}

\usepackage{geometry}
\geometry{textheight=8.5in, textwidth=6in}

%random comment

\newcommand{\cred}[1]{{\color{red}#1}}
\newcommand{\cblue}[1]{{\color{blue}#1}}

\usepackage{hyperref}
\usepackage{geometry}

\def\name{David Baugh}

%pull in the necessary preamble matter for pygments output
\input{pygments.tex}

%% The following metadata will show up in the PDF properties
\hypersetup{
  colorlinks = true,
  urlcolor = blue,
  pdfauthor = {\name},
  pdfkeywords = {CS460 capstone ``computer science'' problem statement},
  pdftitle = {Social Media Joural App Problem Statement},
  pdfsubject = {Problem Statement},
  pdfpagemode = UseNone
}

\begin{document}
\begin{titlepage}
\begin{center}
\vspace*{3cm}
\huge{\textbf{Social Media Journal App}}\par
\vspace{1cm}
\large{Project Sponsored By:}\par
\vspace{3mm}
\normalsize{David Vasquez}\par
\vspace{1cm}
\huge{\textbf{Problem Statement}}\par
\vspace{1cm}
\large{Written by:}\par
\vspace{3mm}
\normalsize{David Baugh}\par
\vspace{25mm}
\large{\textbf{Abstract:}}\par 
\vspace{4mm}
\normalsize{The Social Media Journal App project seeks to develop a social media 
app not unlike Facebook, Instagram, and Twitter. Our group will develop fill the 
niche that exists today where social media can only capture a snippet of your day. 
The Social Media Journal App will allow consumers to express their everyday life 
in the form of journals that will allow for extended word counts and embedded photos.

This project will develop from the ground up starting with the database and ending
 with the graphic design and UI development for the app. The Social Media Journal App
  will be a project that by the end of the year we will have a finished project or a 
  project that we will hopefully continue outside of class.}
\end{center}
\end{titlepage}

\setlength\parindent{4mm}
\section*{Problem Definition}\par
\hspace{4ex}In the world filled with social media and entertainment, lives are 
posted online 140 characters and a picture at a time. No one social media is 
a complete solution for every person and thus leaves some without a way to 
express themselves online.

\section*{Problem Description}\par
\hspace{4ex}Other than a Facebook post, there is no way to express
 yourself online in an elongated way. Instagram only allows for short captions 
 with each photo. Every popular social media app has its perks but they do not 
 fill the niche for every person. The challenge of this project is to create a
 full social media network that will fill one of those niches and allow another 
 avenue for the online community to express themselves. This social media app 
 has to look and feel like a welcome place that people can come and feel comfortable
telling their stories and entries. The next problem is that the Social Media Journal
App group will have to make sure the back end is up to specs for the front UI to be
able to encourage open sharing about people's lives and adventures. To do this, the 
 project will need to develop from the ground up from the database and networking
 and all the way through security and up to the UI. The app will need to be 
innovative and fresh as to garner attention from the public and local communities.
Security and connectivity will be essential with the app as it will contain users'
personal information and person stories. The database and network will need to be
airtight because without a solid foundation, the house will fall and the app will 
fail.  

\section*{Proposed Solution}\par
\hspace{4ex}The Social Media Journal App project will explore new and innovative 
ways of expressing one's self online. The project will focus on the niche of 
journaling in which users will be able to write extended thoughts with embedded 
photos and potentially more. The project wants to combine the ideas of social
media and journalism and form a collection of community's and personal journals 
that allow people to see into other's lives and really understand and enjoy 
other's high points in life journal entry by journal entry. These journal entries
could be described as Twitter posts with the robustness of Facebook and the 
suave simplicity of Instagram. 

On a user's profile, they would have the ability to look through their life within 
their journal entries. For some this would work as a diary; others it would be a 
public journal of their adventures. Users would have access to the communities 
they follow. They would be able to create a new post which would include photos, 
text, and whatever else they want embedded.

Initially the app will consists of individual profiles that allow a person to share 
their own journal entries on but hopefully the idea will be expanded into the thought 
of communities. These communities will allow groups of people or organizations to have 
a place that they can communally place journal entries. Users will be able to follow these 
communities and have the ability to get a glimpse into another's life. Imagine an OSU community where
 students come on and post journal entries about their day at OSU. They could then come 
 back to that community later and see the lives of students along with their own story 
 every day according to the OSU community.  

From the technical side, this project will consist of a full stack development. 
The project will consist of modular but robust code that will be easily edited 
and changed as the project evolves over the project's life time. The database will
be situated on OSU's student databases up to the point when the project extends past
the grasp of capstone. From there, it will up to the group members who stay on the
project where to situate the database from there on. React Native will be used
as it will allow for both Android and IOS development concurrently. Networking and
Security will be developed with professional standards as they are essential to the
projects stability and survival. The UI of the app will be innovative in that it will
be attractive to new users and is comfortable for the consistent users.


\section*{Performance Metrics}\par
\hspace{4ex}Project completion will be judged by the following metrics:
\begin{enumerate}
\item Database:
\begin{itemize}
\item Can the database capable of 10s or 100s of user's information?
\item Will the database be able to be transfer at end of term?
\item Is the database capable of being edited and viewed?
\end{itemize}
\item Networking and Security:
\begin{itemize}
\item Does the network maintain a consistent connection between apps and database?
\item Does the network maintain a secure connection to and from the database?
\item Is there secure account registration?
\item Is there secure log in and log out?
\end{itemize}

\item UI:
\begin{itemize}
\item Can the user create a profile for themselves or a community?
\item Can the user view and follow other profiles?
\item Can the user make journal posts?
\item Can the user make journal posts that contain both text and text-wrapped images?
\end{itemize} 
\end{enumerate}
\end{document}




